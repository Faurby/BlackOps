%Nakket fra ISE project
\usepackage[a4paper]{geometry}
\usepackage[utf8]{inputenc}
\usepackage{
    amsmath, % \begin{align}
    amssymb, % math symbols såsom \rightarrow 
    graphicx, % indsæt billeder
    wrapfig, % indsæt figurer ved siden af tekst
    float, % vælge placering af figures med [H]
    enumitem, % ændre label på enumeration
    fancyhdr, % header og footer
    xcolor, % tekstfarve mm.
    colortbl, % farve tabellers celler
    tabularx, % mere kontrol over tabeller
    listings, % kodebokse
    hyperref, % embed links
    nameref, % referere til mere end bare label-number
} 
\usepackage{tikz}
\usepackage{multicol}
\usepackage[final]{pdfpages}
\usepackage{multirow}

% Sideopsætning
\pagestyle{fancy} % Bred side med header og footer
\fancyhead[LR]{} % Headeren er nu tom
\renewcommand{\headrulewidth}{0pt} % Ingen streg under headeren
\renewcommand{\footrulewidth}{0pt} % Ingen streg under footeren
\setcounter{tocdepth}{1} % Hvor meget skal med i tocs

% Tekstformattering
\setlength{\parindent}{0pt} % Ingen indryk
\linespread{1.2} % Linjeafstand
\setlength{\parskip}{1ex plus 0.5ex minus 0.2ex} % Et lille linjeskip i stedet for indryk

\newcommand{\dashline}{- - - - - - - - - - - - - - - - - - - - - - - - - - - - - - - - - - - - - - - - - - - - - - - - - - - - - - - - - - - -}
%\newcommand{\dashline}{-- \;\;-- \;\;-- \;\;-- \;\;-- \;\;-- \;\;-- \;\;-- \;\;-- \;\;-- \;\;-- \;\;-- \;\;-- \;\;-- \;\;-- \;\;-- \;\;-- \;\;-- \;\;-- \;\;-- \;\;-- \;\;-- \;\;-- \;\;-- \;\;-- \;\;-- \;\;-- \;\;-- \;\;-- \;\;-- \;\;-- \;\;-- \;\;--}


\hypersetup{
    colorlinks,
    linkcolor={black},
    citecolor={black},
    urlcolor={blue!80!black}
}


% Tabeller, afhængig af tabularx package
\newcolumntype{R}{>{\raggedleft\arraybackslash}X}
\newcolumntype{B}{>{\columncolor{mygray}\raggedright}X}
\newcolumntype{L}{>{\raggedright}X}
\newcolumntype{C}{>{\centering\arraybackslash}X}


% Matematik
\renewcommand{\mod}{\textbf{ mod }}
\newcommand{\mmod}{\text{mod }}
\renewcommand{\lg}{\text{lg}}
\renewcommand{\div}{\text{ div }}


% Farver
\definecolor{darkgreen}{HTML}{009900}
\definecolor{dkgreen}{HTML}{21892F}
\definecolor{myblue}{HTML}{3237C0}
\definecolor{mygrey}{rgb}{0.5,0.5,0.5}

% farvet tekst, afhængig af colorx package
\newcommand{\red}[1]{\textcolor{red}{#1}}
\newcommand{\grey}[1]{\textcolor{lightgray}{#1}}
\newcommand{\green}[1]{\textcolor{darkgreen}{#1}}
\newcommand{\greenHL}[1]{\sethlcolor{green}\hl{#1}}
\newcommand{\yellowHL}[1]{\sethlcolor{yellow}\hl{#1}}
\newcommand{\redHL}[1]{\sethlcolor{red}\hl{#1}}


% Til kodebokse, afhængig af listings package
\lstset{frame=single,
  language=Java,
  aboveskip=3mm,
  belowskip=3mm,
  showstringspaces=false,
  columns=flexible,
  basicstyle={\small\ttfamily},
  numbers=left, 
  numberstyle=\tiny\color{mygrey},
  keywordstyle=\color{myblue},
  commentstyle=\color{dkgreen},
  breaklines=true,
  breakatwhitespace=true,
  tabsize=3
}
